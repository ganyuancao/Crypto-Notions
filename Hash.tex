%%%%%%%%%%%%%%%%%%%%%%%%%%%%%%%%%%%%%%%%%%%%%%
% Hash Function
%%%%%%%%%%%%%%%%%%%%%%%%%%%%%%%%%%%%%%%%%%%%%%
\newpage
\section{Hash Function}
%%%%%%%%%%%%%%%%%%%%%%%%%%%%%%%%%%%%%%%%%%%%%%


%%%%%%%%%%%%%%%%%%%%%%%%%%%%%%%%%%%%%%%%%%%%%%
\subsection{Collision Resistance} Let $H: \mathcal{D} \rightarrow \mathcal{R}$ be a hash function. An algorithm $\adv$ is said to be $(t,\varepsilon)$ \textit{collision resistance} (CR) adversary against $H$ if $\adv$ runs in time $t$ with advantage

$$
\advantage{CR}{H}[(\adv)] = \Pr[\mathrm{G}^\text{CR} \Rightarrow 1] = \varepsilon
$$ 

\begin{figure}[!h]
\centering
\begin{codeframe}[colback = white, width=7.25cm, height=4cm]{$\mathrm{G}^{\text{CR}}$}
\begin{pchstack}
\procedure[linenumbering]{$\textbf{procedure} $}{ 
(m, m') \sample \adv(\cdot) \\
\pcif m \neq m' \wedge H(m) = H(m') \pcthen \\
	\t \pcreturn 1 \\
\pcelse \\
	\t \pcreturn 0
}
\end{pchstack}
\end{codeframe}
\caption{Collision Resistance (CR) Game}
\label{fig:cr-game}
\end{figure}

\bigskip
\textit{Remarks:}
\begin{enumerate}
	\item Collision must exist because $|\mathcal{D}| \gg |\mathcal{R}|$.
	\item Fix a hash function $H$, there must be an efficient algorithm $\adv$ that outputs collisions. 
	\item Thus we cannot have a security definition for collision resistance that quantifies over all efficient algorithms $\adv$. 
\end{enumerate}




%%%%%%%%%%%%%%%%%%%%%%%%%%%%%%%%%%%%%%%%%%%%%%
\subsubsection{Second Pre-image Resistance}
Let $H: \mathcal{D} \rightarrow \mathcal{R}$ be a hash function. An algorithm $\adv$ is said to be $(t,\varepsilon)$ \textit{second preimage} (2PRE) adversary against $H$ if $\adv$ runs in time $t$ with advantage  

$$
\advantage{2PRE}{H}[(\adv)] = \Pr[\mathrm{G}^\text{2PRE} \Rightarrow 1] = \varepsilon
$$ 

%%%%%%%%%%%%%%%%%%%%%%%%%%%%%%%%%%%%%%%%%%%%%%
\subsubsection{Pre-image Resistance}
Let $H: \mathcal{D} \rightarrow \mathcal{R}$ be a hash function. An algorithm $\adv$ is said to be $(t,\varepsilon)$ \textit{preimage resistance} (PRE) adversary against $H$ if $\adv$ runs in time $t$ with advantage  

$$
\advantage{PRE}{H}[(\adv)] = \Pr[\mathrm{G}^\text{PRE}(\adv) \Rightarrow 1] = \varepsilon
$$ 

\begin{figure}[!h]
\centering
\begin{tabular}{c c}
\begin{codeframe}[colback = white, width=6.5cm, height=5.25cm]{$\mathrm{G}^{\text{2PRE}}$}
\begin{pchstack}[space=0.5cm]
\procedure[linenumbering]{$\textbf{procedure }$}{ 
m \sample \mathcal{D} \\ 
h \gets H(m) \\
m' \sample \adv(m, h) \\
\pcif m \neq m' \wedge H(m') = h \pcthen \\
	\t \pcreturn 1 \\
\pcelse \\
	\t \pcreturn 0
}
\end{pchstack}
\end{codeframe}

&\hspace{0.3cm} 

\begin{codeframe}[colback = white, width=5.5cm, height=5.25cm]{$\mathrm{G}^{\text{PRE}}$}
\begin{pchstack}[space=0.5cm]
\procedure[linenumbering]{$\textbf{procedure }$}{ 
h \sample \mathcal{R} \\
m \sample \adv(h) \\
\pcif H(m) = h \pcthen \\
\t \pcreturn 1 \\
\pcelse \\
\t \pcreturn 0 
}
\end{pchstack}
\end{codeframe}
\end{tabular}
\caption{\textbf{Left}: Second Preimage Resistance (2PRE) Game. \textbf{Right}: Preimage Resistance (PRE) Game}
\label{fig:pre-game}
\end{figure}




%%%%%%%%%%%%%%%%%%%%%%%%%%%%%%%%%%%%%%%%%%%%%%
\subsubsection{One-wayness}
Let $H: \mathcal{D} \rightarrow \mathcal{R}$ be a hash function. An algorithm $\adv$ is said to be $(t,\varepsilon)$ \textit{one-wayness} (OWF) adversary against $H$ if $\adv$ runs in time $t$ with advantage 

$$
\advantage{OWF}{H}[(\adv)] = \Pr[\mathrm{G}^\text{OWF}(\adv) \Rightarrow 1] = \varepsilon
$$ 

\begin{figure}[!h]
\centering
\begin{codeframe}[colback = white, width=5.5cm, height=5cm]{$\mathrm{G}^{\text{OWF}}$}
\begin{pchstack}[space=0.5cm]
\procedure[linenumbering]{$\textbf{procedure }$}{ 
m \sample \mathcal{D} \\ 
h \gets H(m) \\
m' \sample \adv(h) \\
\pcif H(m') = h \pcthen \\
	\t \pcreturn 1 \\
\pcelse \\
	\t \pcreturn 0
}
\end{pchstack}
\end{codeframe}
\caption{One-wayness (OWF) Game}
\label{fig:owf-game}
\end{figure}


%%%%%%%%%%%%%%%%%%%%%%%%%%%%%%%%%%%%%%%%%%%%%%
\subsubsection{Universal Hashing} A keyed hash funcion $H$ is an $\varepsilon$-bounded \textit{universal hash function} ($\varepsilon$-UHF) if for any adversary $\adv$, the advantage $\advantage{UHF}{H}[(\adv)] \leq \varepsilon$ where 
$$
\advantage{UHF}{H}[(\adv)] = \Pr[\mathrm{G}^\text{UHF}(\adv) \Rightarrow 1]
$$

\begin{figure}[!h]
\centering
\begin{codeframe}[colback = white, width=5.75cm, height=5cm]{$\mathrm{G}^{\text{UHF}}$}
\begin{pchstack}
\procedure[linenumbering]{$\textbf{procedure}$}{ 
K \sample \mathcal{K} \\ 
(m_0, m_1) \sample \adv(\cdot) \\ 
\pcif H(K, m_0) = H(K, m_1) \\ 
	\t\pctab\pctab  \wedge m_0 \neq m_1 \pcthen \\ 
	\t \pcreturn 1 \\
\pcelse \\
	\t \pcreturn 0
}
\end{pchstack}
\end{codeframe}
\caption{UHF Game}
\label{fig:uhf-game}
\end{figure}


\subsubsection{Difference Unpredictable Hashing} A keyed hash function $H$ with digest space $\mathcal{T}$ equipped with a group operation "+", is an $\varepsilon$-bounded \textit{difference unpredictable hashing function} if for any adversary $\adv$, the advantage $\advantage{DUHF}{H}[(\adv)] \leq \varepsilon$ where 
$$
\advantage{DUHF}{H}[(\adv)] = \Pr[\mathrm{G}^\text{DUHF}(\adv) \Rightarrow 1]
$$

\begin{figure}[!h]
\centering
\begin{codeframe}[colback = white, width=6.25cm, height=5cm]
{$\mathrm{G}^{\text{DUHF}}$}
\begin{pchstack}
\procedure[linenumbering]{$\textbf{procedure }$}{ 
K \sample \mathcal{K} \\ 
(m_0, m_1, \delta) \sample \adv(\cdot) \\ 
\pcif H(K, m_0) - H(K, m_1) = \delta \\ 
	\t\pctab\pctab \wedge m_0 \neq m_1 \pcthen \\ 
	\t \pcreturn 1 \\
\pcelse \\
	\t \pcreturn 0
}
\end{pchstack}
\end{codeframe}
\caption{DUHF Game}
\label{fig:duhf-game}
\end{figure}