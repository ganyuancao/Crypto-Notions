\newpage
\section{Public-Key Encryption}

\subsection{LoR-CPA Security}
A public-key encryption scheme  $\Pi$ is said to have $(q,t,\varepsilon)$-\textit{indistinguishibility under chosen plaintext attack with left-or-right oracle} ($\lorcpa)$, if for any adversaries $\adv$ running in time at most $t$ and making at most $q$ encryption queries, the advantage $\advantage{\lorcpa}{\Pi}[(\adv)] \leq \varepsilon$ where 

$$
\advantage{\lorcpa}{\Pi}[(\adv)] = 2 \cdot | \Pr[\mathrm{G}^{\lorcpa}(\adv) \Rightarrow \true ] - \frac{1}{2} | 
$$ 


\begin{figure}[h]
\centering
\begin{codeframe}[colback = white, width=9.5cm, height=4.75cm]{$\mathrm{G}^{\lorcpa}$}

\begin{pchstack}[space=0.5cm]
\begin{pcvstack}[space=0.3cm]
\procedure[linenumbering]{$\textbf{procedure } \textsc{Init}$}{ 
b \sample \bin \\
(PK,SK) \sample \mathcal{K} 
}

\procedure[linenumbering]{$\textbf{procedure } \textsc{Finalize}$}{
 b' \sample \adv^{\enc}(PK) \\
\pcreturn b'= b
}
\end{pcvstack}
\procedure[linenumbering]{$\textbf{Oracle }\enc(M_0, M_1)$}{ 
\pcif |M_0| \neq |M_1| \pcthen \\ 
	\t \pcreturn \bot \\
C \sample \mathcal{E}_{PK}(M_b) \\
 \pcreturn C
}
\end{pchstack}
\end{codeframe}
\caption{LoR-CPA Game for a public-key encryption scheme $\Pi$. }
\label{fig:lorcpa-game-pkc}
\end{figure}



\subsection{RoR-CPA Security}
A public-key encryption scheme  $\Pi$ is said to have $(q,t,\varepsilon)$-\textit{indistinguishibility under chosen plaintext attack with real-or-random oracle} ($\rorcpa)$, if for any adversaries $\adv$ running in time at most $t$ and making at most $q$ encryption queries, the advantage $\advantage{\rorcpa}{\Pi}[(\adv)] \leq \varepsilon$ where 

$$
\advantage{\rorcpa}{\Pi}[(\adv)] = \abs{\Pr[\mathrm{G}^{\rorcpa\textrm-0}(\adv)]  - \Pr[\mathrm{G}^{\rorcpa\textrm-1}(\adv)]
} 
$$

\begin{figure}[!h]
    \centering
    \begin{codeframe}[colback = white, width=5.25cm, height=6cm]{$\mathrm{G}^{\rorcpa\textrm-0}$}
        \begin{pcvstack}[space=0.3cm]
            \procedure[linenumbering, width=4cm]{$\textbf{procedure } \textsc{Init} $}{
                (PK,SK) \sample \mathcal{K} \\ 
                \mathcal{Q} \gets \emptyset  
            }
    
            \procedure[linenumbering, width=4cm]{$\textbf{procedure } \textsc{Enc}(M)$}{
                \pcif M \in \mathcal{Q} \pcthen \\
                    \t \pcreturn \bot \\ 
                C \gets \mathcal{E}_{PK}(M) \\ 
                \mathcal{Q} \gets \mathcal{Q} \cup \{M\} \\ 
                \pcreturn C
            }
        \end{pcvstack}
    \end{codeframe}
    \hspace{0.5cm}
    \begin{codeframe}[colback = white, width=5.25cm, height=6cm]{$\mathrm{G}^{\rorcpa\textrm-1}$}
        \begin{pcvstack}[space=0.5cm]
             \procedure[linenumbering, width=4cm]{$\textbf{procedure } \textsc{Init} $}{
                \mathcal{Q} \gets \emptyset  
            }
            
            \procedure[linenumbering, width=4cm]{$\textbf{procedure } \textsc{Enc}(M)$}{
                 \pcif M \in \mathcal{Q} \pcthen \\
                    \t \pcreturn \bot \\ 
                C \sample \bin^{|M|} \\
                \mathcal{Q} \gets \mathcal{Q} \cup \{M\} \\ 
                \pcreturn C
            }
        \end{pcvstack}
    \end{codeframe}
\caption{$\rorcpa$ game for a SE scheme $\Pi$}
\label{fig:rorcpa-game-pkc}
\end{figure}




%%%%%%%%%%%%%%%%%%%%%%%%%%%%%%%%%%%%%%%%%%%%%%
\subsection{LoR-CCA Security}
A public-key encryption scheme  $\Pi$ is defined to be $(q_e, q_d, t,\varepsilon)$-\textit{indistinguishibility under chosen ciphertext attack} secure ($\indcca)$, if for any adversaries $\adv$ running in time at most $t$ and making at most $q_e$ encryption queries to oracle $\enc$ and at most $q_d$ decryption queries to oracle $\dec$, the advantage $\advantage{\indcpa}{\textsf{SE}}[(\adv)] \leq \varepsilon$.

$$
\advantage{\lorcca}{\textsf{SE}}[(\adv)] = 2 \cdot | \mathrm{G}^{\lorcca}(\adv) \Rightarrow \true ] - \frac{1}{2} | 
$$ 


\begin{figure}[!h]
\centering
\begin{codeframe}[colback = white, width=9.75cm, height=6.5cm]{$\mathrm{G}^{\lorcca}$}
\begin{pchstack}[space=0.5cm]
\begin{pcvstack}[space=0.3cm]
\procedure[linenumbering,  width=3.5cm]{$\textbf{procedure } \textsc{Init}$}{ 
b \sample \bin \\
(PK,SK) \sample \mathcal{K} \\
\mathcal{Q}_d \gets \emptyset
}

\procedure[linenumbering, width=3.5cm]{$\textbf{Oracle }\dec(N,C)$}{ 
\pcif (N,C) \in \mathcal{Q} \pcthen \\
    \t \pcreturn \bot \\
M \gets \mathcal{D}_{SK}(C) \\
\mathcal{Q}_d \gets \mathcal{Q}_d \cup \{(N,C)\} \\ 
 \pcreturn M
}
\end{pcvstack}

\begin{pcvstack}[space=0.3cm]
\procedure[linenumbering, width=4cm]{$\textbf{Oracle }\enc(N,M_0, M_1)$}{ 
\pcif |M_0| \neq |M_1| \pcthen \\ 
	\t \pcreturn \bot \\
C \sample \mathcal{E}_{PK}(M_b) \\
 \pcreturn C
}
\procedure[linenumbering, width=4cm]{$\textbf{procedure } \textsc{Finalize}$}{
 b' \sample \adv^{\enc, \dec}(\cdot) \\
\pcreturn b'= b
}
\end{pcvstack}

\end{pchstack}
\end{codeframe}
\caption{LoR-CPA Game for a \textsf{SE} scheme $\Pi$. }
\label{fig:lorcca-game-pkc}
\end{figure}



\subsection{RoR-CCA Security}
A public-key encryption scheme  $\Pi$ is said to have $(q,t,\varepsilon)$-\textit{indistinguishibility under chosen plaintext attack with real-or-random oracle} ($\rorcpa)$, if for any adversaries $\adv$ running in time at most $t$ and making at most $q$ encryption queries, the advantage $\advantage{\rorcpa}{\Pi}[(\adv)] \leq \varepsilon$ where 

$$
\advantage{\rorcpa}{\Pi}[(\adv)] = \abs{\Pr[\mathrm{G}^{\rorcpa\textrm-0}(\adv)]  - \Pr[\mathrm{G}^{\rorcpa\textrm-1}(\adv)]
} 
$$

\begin{figure}[!htp]
    \centering
    \begin{codeframe}[colback = white, width=5.25cm, height=9.5cm]{$\mathrm{G}^{\rorcca\textrm-0}$}
        \begin{pcvstack}[space=0.3cm]
            \procedure[linenumbering, width=3.5cm]{$\textbf{procedure } \textsc{Init} $}{
                K \sample \mathcal{K} \\ 
                \mathcal{Q}_e, \mathcal{Q}_d \gets \emptyset  
            }
    
            \procedure[linenumbering, width=3.5cm]{$\textbf{procedure } \textsc{Enc}(M)$}{
                \pcif M \in \mathcal{Q}_e \pcthen \\
                    \t \pcreturn \bot \\
                C \gets \mathcal{E}_{PK}(M) \\ 
                \mathcal{Q}_e \gets \mathcal{Q}_e \cup \{M\} \\ 
                \mathcal{Q}_d \gets \mathcal{Q}_d \cup \{C\} \\ 
                \pcreturn C
            }
            \procedure[linenumbering, width=3.5cm]{$\textbf{procedure } \textsc{Dec}(N,C)$}{
                \pcif C \in \mathcal{Q}_d \pcthen \\
                    \t \pcreturn \bot \\ 
                M \gets \mathcal{D}_{SK}(C) \\ 
                \pcreturn C
            }
        \end{pcvstack}
    \end{codeframe}
    \hspace{0.5cm}
    \begin{codeframe}[colback = white, width=5.25cm, height=9.5cm]{$\mathrm{G}^{\rorcca\textrm-1}$}
        \begin{pcvstack}[space=0.3cm]
            \procedure[linenumbering, width=3.5cm]{$\textbf{procedure } \textsc{Init} $}{
                \mathcal{Q}_e, \mathcal{Q}_d \gets \emptyset  
            }
            
            \procedure[linenumbering, width=3.5cm]{$\textbf{procedure } \textsc{Enc}(N,M)$}{
                \pcif M \in \mathcal{Q}_e \pcthen \\
                    \t \pcreturn \bot \\ 
                C \sample \bin^{|M|} \\
                \pcreturn C
            }
            \procedure[linenumbering, width=3.5cm]{$\textbf{procedure } \textsc{Dec}(C)$}{
                C \gets \mathcal{D}_{SK}(C) \\ 
                \pcreturn C
            }
        \end{pcvstack}
    \end{codeframe}
\caption{$\rorcpa$ game for a public-key encryption scheme $\Pi$}
\label{fig:rorcca-game-pkc}
\end{figure}
